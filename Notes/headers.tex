\usefonttheme[onlymath]{serif}
\usepackage[utf8]{inputenc}
\usepackage[T1]{fontenc}
\usepackage{lmodern} % important!

\usepackage{hyperref} % probably already included by beamer package
\hypersetup{pdfauthor={Stephen Becker},
    pdftitle={},
    colorlinks=true,
%    citecolor=MidnightBlue, % requires \usepackage[usenames,dvipsnames,svgnames]{xcolor}
%    urlcolor=Bittersweet,
}


% === Common packages we need sometimes ===
\usepackage{amssymb,amsfonts}
\usepackage{amsmath}
%\usepackage{amsthm} % unnecessary given beamer's theorem class
%\usepackage{color}
%\usepackage{rotating}
%\usepackage{multirow}
%\usepackage{fancybox}
%\usepackage[usenames,dvipsnames,svgnames]{xcolor} % use like \textcolor{red}{..} or {\color{red} ...}
%\usepackage[usenames]{color} % This has ``option clash'' ???
\usepackage{graphicx} % clashes with color, xcolor packages
%\usepackage{booktabs}
%\usepackage{url}
\usepackage{enumitem} % use like \begin{itemize}[noitemsep,nosep,itemindent=-2ex,label=$\circ$]
\usepackage{microtype}	% better font kerning. Not necessary.

%\usepackage[ruled,vlined]{algorithm2e} \usepackage{algorithmic}
%\usepackage{algorithm,algpseudocode} % algorithmicx package -- my preferred

    % create an overlay:
    %\pause

% === Adjust style of slides ===
\mode<presentation>{
    % This means ``hidden'' slides are slightly visible
    % Default is ``invisible'', but you can do
    % transparent=85  (85% is the default)
    % or ``dynamic'', meaning way uncovered text is lighter
%    \setbeamercovered{dynamic}
    \setbeamercovered{transparent=5}
%    \setbeamercovered{transparent}
}
%Themes with side outline: Berkeley, Goettingen, Hannover, Marburg, PaloAlto, 
%\usetheme{Goettingen}
%\usetheme{Hannover}
%\usetheme{Szeged}
\usetheme{Boadilla}
%Themes with top outline: Antibes, Copenhagen, Dresden, Frankfurt, Ilmenau, JuanLesPins, Luebeck, Malmoe, Montpellier, Szeged, Warsaw
% Colors: dolphin, dove (fewer colors), lily, default
% 95% of Boadilla is these lines:
%\usecolortheme{lily}
%\usecolortheme{rose}
%\useinnertheme[shadow]{rounded}
%\usecolortheme{dolphin} % OK
% %\useoutertheme{infolines}
% %\setbeamertemplate{footline}[default]
%\useoutertheme{infolines} % defines colors...
%\setbeamertemplate{headline}[default]
%\setbeamertemplate{footline}[frame number] % this works
% %\setbeamertemplate{footline}[page number]{}  % removes bottom part completely, leaves page numbers

\setbeamertemplate{navigation symbols}{} % remove navigation symbols

%  the rounded inner theme uses a weird ball for itemize which I don't like:
\makeatletter   % for ``@'' character...
%\setbeamertemplate{items}[ball]  % or circle...
\setbeamertemplate{items}[circle]  % or circle...
\makeatother

%\AtBeginSection[]
%\AtBeginSubsection[]
%{\begin{frame}
%        \frametitle{Outline}
%        \small
%        %\tableofcontents[currentsection,hidesubsections]
%        %        \tableofcontents[currentsection,%]%,% % normal one
%        %        currentsubsection]
%        %hideothersubsections,
%        %sectionstyle=show/shaded,
%        %%sectionstyle=show/hide,
%        %subsectionstyle=show/shaded,
%        %]
%        \tableofcontents[currentsection]
%        %\tableofcontents[currentsubsection]
%        \normalsize
%        \addtocounter{framenumber}{-1}
%\end{frame}}



% === Macros ===
\usepackage{xspace}
\newcommand{\thebook}{Shalev-Shwartz and Ben-David\xspace}

\newcommand{\R}{\mathbb{R}}
\newcommand{\E}{\mathbb{E}}
\DeclareMathOperator*{\EE}{\mathbb{E}} % this way allows it to have nice subscript
\renewcommand{\H}{\mathcal{H}}
\renewcommand{\phi}{\varphi}
\newcommand{\bx}{\mathbf{x}}  % b for bold
\newcommand{\x}{\bx}
\newcommand{\bX}{\mathbf{X}}
\newcommand{\X}{\bX}
\newcommand{\by}{\mathbf{y}} 
\newcommand{\y}{\by}
\newcommand{\bz}{\mathbf{z}} 
\newcommand{\z}{\bz}
\newcommand{\ba}{\mathbf{a}}
\newcommand{\bv}{\mathbf{v}}
\newcommand{\bu}{\mathbf{u}}
\newcommand{\bw}{\mathbf{w}}
\newcommand{\w}{\bw}
%\newcommand{\bsigma}{\mathbf{\sigma}}  % b for bold.  This doesn't work
\newcommand{\bsigma}{{\boldsymbol{\sigma}}}  % b for bold
\newcommand{\cX}{\mathcal{X}} % c for caligraphic
\newcommand{\cY}{\mathcal{Y}}
\newcommand{\cD}{\mathcal{D}}
\newcommand{\cO}{\mathcal{O}} % for Order
\newcommand{\order}{\cO} % already defined
\newcommand{\cN}{\mathcal{N}} % for Normal
\newcommand{\loss}{\ell} 	% loss function
\newcommand{\risk}{L} 		% risk, generic
\newcommand{\riskD}{\risk_{\cD}} 	% true risk
\newcommand{\riskEmp}{\widehat{\risk}}
\newcommand{\riskS}{\riskEmp_S}
\newcommand{\Rad}{\mathfrak{R}} % expected Rademacher complexity
\newcommand{\RadEmp}{\widehat{\Rad}} % empirical Rademacher complexity
\DeclareMathOperator{\VC}{VCdim}
\DeclareMathOperator*{\argmin}{argmin}
\newcommand{\defeq}{\stackrel{\text{\tiny def}}{=}} 
\newcommand{\simiid}{\stackrel{\text{\tiny iid}}{\sim}} 
\newcommand{\<}{\langle}  %  useful!!!
\renewcommand{\>}{\rangle}
\newcommand{\iprod}[2]{\left\langle #1 , #2 \right\rangle}
\newcommand{\algo}{\texttt{A}}
